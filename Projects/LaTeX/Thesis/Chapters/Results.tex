\chapter{Results}
This work's implementation of the proposed platform is described in section \ref{sec:proposed-platform-implementation}. Its performance metrics, such as its latency, throughput, power, and energy consumption, are compared to the available alternative technologies and FPGA architectures. AlexNet is the selected CNN to be used as a benchmark for the various technologies compared.

\section{Specifications of the Compared Platforms}
The proposed platform is compared with an Intel i7 4710MQ CPU, an NVIDIA RTX 2060 Super 8GB GPU, a Xilinx CHaiDNN implementation, and a Xilinx DPU implementation. All FPGA implementations, including this work's platform, use the Xilinx ZCU102 Evaluation board.

\subsection{AMD Ryzen 5 1400}
The AMD Ryzen 5 1400 CPU \cite{AMD-Ryzen-5-1400-Processor}, released in 2017, is a desktop processor targeted for mid-range productivity computers. Its specifications are presented in table \ref{tab:AMD-Ryzen-5-1400-specs}.

\begin{table}[H]
	\caption{AMD Ryzen 5 1400 processor specifications}
	\label{tab:AMD-Ryzen-5-1400-specs}
	\centering
	\begin{tabular}{ll}
		\toprule
		\textbf{Cores / Threads} & 4/8\\
		\textbf{Max Turbo Frequency} & 3.4GHz\\
		\textbf{TDP} & 65W\\
		\textbf{Max Memory Bandwidth} & 42.7GB/s\\
		\textbf{Lithography} & 14nm\\
		\bottomrule\\
	\end{tabular}
\end{table}

% \subsection{Intel i7 4710MQ}
% The Intel i7 4710MQ CPU \cite{Intel-i7-4710MQ-Processor}, released in 2014, is a mobile processor targeted for high-performance laptops. Its specifications are presented in table \ref{tab:Intel-i7-4710MQ-specs}.

% \begin{table}[H]
% 	\caption{Intel i7 4710MQ processor specifications}
% 	\label{tab:Intel-i7-4710MQ-specs}
% 	\centering
% 	\begin{tabular}{p{2cm} p{3cm} p{1cm} p{3cm} p{3cm}}
% 		\toprule
% 		\textbf{Cores / Threads} & \textbf{Max Turbo Frequency} & \textbf{TDP} & \textbf{Max Memory Bandwidth} & \textbf{Lithography}\\
% 		\midrule
% 			4/8 & 3.5GHz & 47W & 25.6GB/s & 22nm\\
% 		\bottomrule\\
% 	\end{tabular}
% \end{table}

\subsection{NVIDIA RTX-2060 Super 8GB}
The NVIDIA RTX-2060 Super \cite{NVIDIA-RTX-2060-Super}, released in 2019, is a desktop GPU, and while targeted for raytraced gaming, it is also suitable for CNN inferencing due to its large and high-bandwidth memory. Its specifications are presented in table \ref{tab:NVIDIA-RTX-2060-Super-specs}.

\begin{table}[H]
	\caption{NVIDIA RTX 2060 Super specifications}
	\label{tab:NVIDIA-RTX-2060-Super-specs}
	\centering
	\begin{tabular}{ll}
		\toprule
		\textbf{CUDA Cores} & 2176\\
		\textbf{GPU Memory} & 8GB GDDR6\\
		\textbf{Boost Clock} & 1650 MHz\\
		\textbf{Memory Interface} & 256-bit\\
		\textbf{Memory Bandwidth} & 448GB/s\\
		\textbf{Power Consumption} & 175W\\
		\bottomrule\\
	\end{tabular}
\end{table}

\subsection{Xilinx CHaiDNN}
The Xilinx CHaiDNN accelerator library, presented in section \ref{sec:Xilinx-CHaiDNN}, was implemented for this work's comparisons. The resource utilization for its implementation is depicted in table \ref{tab:CHaiDNN-resource-usage}.

\begin{table}[H]
	\caption{Xilinx CHaiDNN resource usage}
	\label{tab:CHaiDNN-resource-usage}
	\centering
	\begin{tabular}{ll}
		\toprule
		\textbf{Clock Frequency} & 250/500 MHz\\
		\textbf{LUT Usage} & 59.1\%\\
		\textbf{LUTRAM Usage} & -\\
		\textbf{FF Usage} & 27.66\%\\
		\textbf{BRAM Usage} & 74.12\%\\
		\textbf{DSP Usage} & 53.65\%\\
		\textbf{BUFG Usage} & -\\
		\bottomrule\\
	\end{tabular}
\end{table}

% \subsection{Xilinx DPU}
% The Xilinx DPU accelerator library, presented in section \ref{sec:Xilinx-DPU}, was implemented for this work's comparisons. The resource utilization for its implementation is depicted in table \ref{tab:DPU-resource-usage}.

% \begin{table}[H]
% 	\caption{Xilinx DPU resource usage}
% 	\label{tab:DPU-resource-usage}
% 	\centering
% 	\begin{tabular}{ll}
% 		\toprule
% 		\textbf{Clock Frequency (MHz)} & 300/600\\
% 		\textbf{LUT Usage (\%)} & 7.34\\
% 		\textbf{LUTRAM Usage (\%)} & 2.05\\
% 		\textbf{FF Usage (\%)} & 4.03\\
% 		\textbf{BRAM Usage (\%)} & 7.51\\
% 		\textbf{DSP Usage (\%)} & 1.9\\
% 		\textbf{BUFG (\%)} & 0.25\\
% 		\bottomrule\\
% 	\end{tabular}
% \end{table}

\section{Proposed Platform}
\label{sec:proposed-platform-implementation}
This work's proposed platform can be implemented using several data types and accelerators. For this comparison, it was implemented based on AlexNet's requirements and characteristics, presented in Robustness Analysis chapter \ref{Chapter-Robustness-Analysis}. Both parameters and activations are represented as 8-bit fixed-points; hence, the accelerators use the same data type. There is a single accelerator instance per layer type, using their high-performance versions as presented in section \ref{sec:Platform-Accelerator-Architectures}. The serial execution scheduler (see section \ref{sec:Scheduler-Strategies}) was selected because the implemented accelerators do not support layer pipelining, and there are not multiple instances per layer type.

The resource utilization for implementing the aforementioned configuration of the proposed platform is depicted in table \ref{tab:Proposed-platform-resource-usage}.

\begin{table}[H]
	\caption{Proposed platform resource usage}
	\label{tab:Proposed-platform-resource-usage}
	\centering
	\begin{tabular}{ll}
		\toprule
		\textbf{Clock Frequency (MHz)} & 300MHz\\
		\textbf{LUT Usage} & 7.34\%\\
		\textbf{LUTRAM Usage} & 2.05\%\\
		\textbf{FF Usage} & 4.03\%\\
		\textbf{BRAM Usage} & 7.51\%\\
		\textbf{DSP Usage} & 1.9\%\\
		\textbf{BUFG (\%)} & 0.25\%\\
		\bottomrule\\
	\end{tabular}
\end{table}

\section{Power Consumption}
Power consumption is defined as the energy consumed per unit time for accomplishing a specific task, from a chemical reaction and lifting materials using a crane, to emitting light through a light bulb and inferencing CNNs on electronic hardware. Average power consumption is always preferred to be as low as possible to increase the system's energy efficiency, minimizing energy losses. In addition, low power consumption leads to simpler system designs and lower building costs. It is usually measured in Watts (w) or kiloWatts (kW).

\section{Energy Consumption}
Energy consumption is defined as the energy required for accomplishing a specific task in a specific time amount. It can be calculated as $Energy = Power * Time$, where $Power$ is the required power, and $Time$ is the required time for accomplishing the task. Energy consumption is also preferred to be as low as possible while accomplishing the given task within the time constraints, to minimize the operational costs. It is usually measured in Joule (J) or kiloJoule (kJ).

\section{Throughput and Latency}
Throughput, defined in equation \ref{eqn:throughput}, is the number of tasks that can be accomplished in a unit time. It is preferred to be as high as possible to generate as much work as possible in the unit time.

Latency, defined in equation \ref{eqn:latency}, is the time required for accomplishing a single task. It is preferred to be as low as possible to finish tasks as quickly as possible from the time they are issued.

\section{Final Performance}
The comparisons were conducted using the same dataset and AlexNet hyper-parameters across all technologies. The CPU and GPU use floating-point arithmetic for their parameters and activations, while CHaiDNN uses 6 and 8-bit quantization, DPU uses 8-bit quantization and pruning, and the proposed platform implementation, TUCNNPU, uses 8-bit fixed-point arithmetic.

Table \ref{tab:Performance-results} depicts the comparison results of every technology.

\begin{table}[H]
	\caption{Performance results}
	\label{tab:Performance-results}
	\centering
	\begin{tabular}{lllll}
		\toprule
		& \textbf{CPU} & \textbf{GPU} & \textbf{CHaiDNN} & \textbf{TUCNNPU}\\
		\midrule
			\textbf{Clock Frequency (MHz)} & 3500 & 1650 & 250/500 & 300\\
			\textbf{Throughput (Images/s)} & 99.75 & 5784.6 & 10.07 & 0.0927\\
			\textbf{Throughput Speedup} & 1x & 57.9909x & 0.1009x & 0.0009x\\
			\textbf{Latency (s)} & 0.01003 & 0.00086 & 0.0993 & 10.783\\
			\textbf{Latency Speedup} & 1x & 11.6627x & 0.101x & 0.00093x\\
			% \textbf{GFLOPS} & - & - & - & -\\
			\textbf{Total On-Chip Power (Watt)} & 65 & 175 & 19.3 & 4.559\\
			\textbf{Power Efficiency} & 1x & 0.3714x & 3.367x & 14.257x\\
			\textbf{Energy Cons./Image (Joule)} & 0.6519 & 0.1505 & 1.9164 & 49.1596\\
			\textbf{Energy Efficiency} & 1x & 4.3315x & 0.3401x & 0.0132x\\
			\textbf{Images/Joule} & 1.533 & 6.644 & 0.521 & 0.02\\
		\bottomrule\\
	\end{tabular}
\end{table}
